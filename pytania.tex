\documentclass[11pt,leqno]{amsart}
\usepackage[T1]{fontenc}
\usepackage[utf8]{inputenc}
\usepackage{amssymb}
\usepackage{hyperref}
\usepackage{cleveref}
\usepackage{url}
%\usepackage[polish]{babel}
%\usepackage{a4wide}
%\usepackage{dsfont}
\frenchspacing
\newtheorem{pyt}{Pytanie}
\newtheorem{obs}{Obserwacja}
\newtheorem{thm}{Theorem}[section]
%\newcommand{\thmautorefname}{theorem}
\newtheorem{lem}[thm]{Lemma}
%\newcommand{\lemautorefname}{lemma}
\newtheorem{cor}[thm]{Corollary}
%\newcommand{\corautorefname}{corollary}
\newtheorem{prop}[thm]{Proposition}
%\newcommand{\propautorefname}{proposition}
\newtheorem{fct}[thm]{Fact}
%\newcommand{\fctautorefname}{fact}
\theoremstyle{remark}
\newtheorem*{rem}{Remark}
\theoremstyle{definition}
\newtheorem*{dfn}{Definition}
\newtheorem*{clm}{Claim}
%\newcommand{\dfnautorefname}{definition}
\newtheorem{ex}[thm]{Example}
%\newcommand{\exautorefname}{example}
\newcommand{\mathilde}[1]{\widetilde{\mathcal #1}}
%\newcommand{\unt}{\mathds{1}}
\newcommand{\supp}{\operatorname{supp}}
\newcommand{\Spec}{\operatorname{Spec}}
\newcommand{\rng}{\operatorname{rng}}
\newcommand{\mon}{\mathfrak C}
\newcommand{\proves}{\vdash}
\newcommand{\liff}{\leftrightarrow}
\newcommand{\limplies}{\rightarrow}
\newcommand{\bigland}{\bigwedge}
\newcommand{\biglor}{\bigvee}
\newcommand{\Aut}{\operatorname{Aut}}
\newcommand{\Autf}[1]{\operatorname{Aut\,f}_{#1}}
\newcommand{\tp}{\operatorname{tp}}
\newcommand{\Th}{\operatorname{Th}}
\newcommand{\Stab}{\operatorname{Stab}}
%\newcommand{citethm}[1]{Theorem {\ref{#1}}}
%\newcommand{citelem}[1]{Lemma \ref{#1}}
%\newcommand{citecor}[1]{Corollary \ref{#1}}
%\newcommand{citeprop}[1]{Proposition \ref{#1}}
%\newcommand{citefct}[1]{Fact \ref{#1}}
%\newcommand{citedfn}[1]{Definition \ref{#1}}
%\newcommand{citeex}[1]{Example \ref{#1}}
\begin{document}

\author{Tomasz Rzepecki}
\title{Pytania}
\begin{pyt}
Czy dla $\overline a\subseteq \overline b$ (lub nawet $\overline a\subseteq\overline n$) mamy $E^{\overline a}_L\leq_B E^{\overline b}_L$? Lub $F^{\overline a}_L\leq F^{\overline b}_L$?

Podobnie dla innych orbitalnych relacji równoważności.
\end{pyt}

\begin{pyt}
Czy $Gal_L(T)=Gal_L(T^{\textrm{eq}})$?
\end{pyt}

\begin{pyt}
Czy są takie $T,X$, że $E^M_L\vert_X$ ma moc borelowską większą niż $\mathbb E_0$, ale mniejszą niż $\ell^\infty$?
\end{pyt}

\begin{pyt}
Czy są takie niezmiennicze relacje równoważności, które w ramach typu (typu KP?) mają moce jak wyżej? Jakie dokładnie?

Jakie relacje orbitalne?
\end{pyt}

\begin{pyt}
Czy istnieje taka $T$, że istnieje podgrupa $\Autf {KP}$ skończonego indeksu?
\end{pyt}

\begin{pyt}
Czy gdy $G$ działa na $A$, to składowe spójne $A$ są $G$-niezmiennicze? (Tak w przypadku NIP, ale co w przeciwnym wypadku?)
\end{pyt}

\begin{pyt}
Ciekawe przykłady nieorbitalnych relacji o nieskończenie wielu klasach?
\end{pyt}

\begin{pyt}
Czy można podzielić typ Lascara, każdą klasę na skończenie wiele częsci, tak żeby uzyskać gładką relację?
\end{pyt}

\begin{pyt}

\end{pyt}
\end{document}
