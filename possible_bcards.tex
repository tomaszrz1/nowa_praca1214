% !TeX encoding = UTF-8
% !TeX spellcheck = en_GB
\documentclass[final,a4paper,12pt]{amsart}
\setlength{\emergencystretch}{2em}

\usepackage[marginratio=1:1]{geometry}
\date{\today}

\usepackage[T1]{fontenc}
\usepackage{ae,aecompl}
\usepackage[utf8]{inputenc}
\usepackage[british]{babel}
\usepackage{csquotes}
\usepackage[draft=false]{hyperref}
\usepackage{xcolor}
%\usepackage{imakeidx}
%\usepackage[notref,notcite]{showkeys}
%\usepackage{showidx}
\usepackage[modulo,pagewise,mathlines]{lineno}
\linenumbers
%\usepackage[amsmath,amsthm]{ntheorem}
\usepackage{amsfonts}
\usepackage{amssymb}
\usepackage{amsmath}
\usepackage{amsthm}
\usepackage{tikz-cd}
\usepackage{float}
\usepackage{graphicx}
\usepackage{tikz}
\usepackage{thmtools}
\usepackage{thm-restate}
\usepackage{mathrsfs}
\usepackage{enumitem}
\usepackage{mathtools}
%\usepackage[backend=biber,
%url=false,
%isbn=false,
%backref=true,
%citestyle=alphabetic,
%bibstyle=alphabetic,
%autocite=inline,
%sorting=ydnt,]{biblatex}
%
%\addbibresource{bibliography.bib}



\newcommand{\fC}{{\mathfrak C}}
\newcommand{\cM}{{\mathcal M}}
\newcommand{\cN}{{\mathcal N}}
\newcommand{\cB}{{\mathcal B}}
\newcommand{\bN}{{\mathbf{N}}}
\newcommand{\bR}{{\mathbf{R}}}
\newcommand{\bZ}{{\mathbf{Z}}}
\newcommand{\bQ}{{\mathbf{Q}}}
\newcommand{\cA}{{\mathcal A}}
\newcommand{\topo}{{\mathrm{top}}}
\newcommand\Lasc{{\mathrm{L}}}
\newcommand\KP{{\mathrm{KP}}}
\newcommand\Sh{{\mathrm{Sh}}}
\newcommand{\restr}{\mathord{\upharpoonright}}
\newcommand{\EZ}{\mathrel{ { {\mathbf E}_0 } } }
\newcommand{\Er}{\mathrel{E}}
\newcommand{\Fr}{\mathrel{F}}
\newcommand{\lang}{{\mathcal L}}
\newcommand{\catg}{{\mathcal C}}
\newcommand{\powerset}{{\mathcal P}}
\newcommand{\liff}{\mathrel{\leftrightarrow}}
\newcommand{\limplies}{\mathrel{\rightarrow}}
\newcommand{\bigland}{\bigwedge}
\newcommand{\biglor}{\bigvee}
\newcommand{\proves}{\vdash}
\newcommand{\Rr}{\mathrel{R}}
\newcommand{\Fin}{\mathrm{Fin}}
\newcommand{\an}{\mathrm{an}}
\DeclareMathOperator{\SO}{{SO}}
\DeclareMathOperator{\GL}{{GL}}
\DeclareMathOperator{\st}{{st}}
\DeclareMathOperator{\cl}{{cl}}
\DeclareMathOperator{\tp}{{tp}}
\DeclareMathOperator{\acl}{{acl}}
\DeclareMathOperator{\dcl}{{dcl}}
\DeclareMathOperator{\Th}{{Th}}
\DeclareMathOperator{\Gal}{{Gal}}
\DeclareMathOperator{\Baire}{{Baire}}
\DeclareMathOperator{\Id}{{Id}}
\DeclareMathOperator{\id}{{id}}
\DeclareMathOperator{\Aut}{{Aut}}
\DeclareMathOperator{\Homeo}{{Homeo}}
\DeclareMathOperator{\Autf}{{Aut\mkern 0.5\thinmuskip f}}
\DeclareMathOperator{\CLO}{{CLO}}
\DeclareMathOperator{\dom}{{dom}}
\DeclareMathOperator{\Core}{{Core}}
\DeclareMathOperator{\Stab}{{Stab}}
\DeclareMathOperator{\SL}{{SL}}
\DeclareMathOperator{\Souslin}{{\mathcal A}}
\let\unlhd\trianglelefteq




\newtheorem{mainthm}{Main Theorem}
\renewcommand*{\themainthm}{\Alph{mainthm}}
\newtheorem{thm}{Theorem}[section]
\newtheorem{conj}[thm]{Conjecture}
\newtheorem{ques}[thm]{Question}
\newtheorem{problem}[thm]{Problem}
\newtheorem{lem}[thm]{Lemma}
\newtheorem{fct}[thm]{Fact}
\newtheorem{cor}[thm]{Corollary}
\newtheorem{prop}[thm]{Proposition}
\newtheorem{qu}[thm]{Question}
\newtheorem{con}[thm]{Conjecture}

\theoremstyle{remark}
\newtheorem{rem}[thm]{Remark}
\theoremstyle{definition}
\newtheorem{dfn}[thm]{Definition}
\newtheorem*{sbclm}{Subclaim}
\newtheorem*{clm*}{Claim}
\newtheorem{ex}[thm]{Example}
\newcounter{claimcounter}[thm]
\newenvironment{clm}{\stepcounter{claimcounter}{\noindent {\textbf{Claim}} \theclaimcounter:}}{}
\newenvironment{clmproof}[1][\proofname]{\proof[#1]\renewcommand{\qedsymbol}{$\square$(claim)}}{\endproof}
\newenvironment{sbclmproof}[1][\proofname]{\proof[#1]\renewcommand{\qedsymbol}{$\square$(subclaim)}}{\endproof}

\newcommand{\xqed}[1]{%
	\leavevmode\unskip\penalty9999 \hbox{}\nobreak\hfill
	\quad\hbox{\ensuremath{#1}}}

\let\Gamma\varGamma
\let\Delta\varDelta
\let\Theta\varTheta
\let\Lambda\varLambda
\let\Xi\varXi
\let\Pi\varPi
\let\Sigma\varSigma
\let\Upsilon\varUpsilon
\let\Phi\varPhi
\let\Psi\varPsi
\let\Omega\varOmega
\let\phi\varphi




\begin{document}
	
	\address{
		Instytut Matematyczny, Uniwersytet Wrocławski,
		pl. Grunwaldzki 2/4, 50-384 Wrocław, Poland
	}
	
	
	
	\author{Tomasz Rzepecki}
	\email[T.\ Rzepecki]{tomasz.rzepecki@math.uni.wroc.pl}
	
	
	
	\section{Borel cardinalities}
	We've established that it makes sense to talk about the Borel cardinality of a bounded, Borel equivalence relation on $\fC$.
	
	\begin{con}[Conjecture 2, \cite{KPS13}]
		Any non-smooth $K_\sigma$ equivalence relation can be represented, for some theory $T$, by some $\equiv_{\Lasc}\restr_X$, where $X$ is a single $\equiv_{\KP}$-class.
	\end{con}
	
	\begin{prop}
		Suppose $S_\alpha(\emptyset)$ is uncountable (this is always the case for countably infinite $\alpha$) and that $M$ is a countable atomic model (this happens, for example, if we take for $T$ the theory of real closed fields, for $M$ the algebraic reals and $\alpha=1$). Then for any Borel equivalence relation $F$ on some uncountable standard Borel space $X$, there is a bounded equivalence relation $E$ such that $E^M\sim_B F$ (even more: $S(M)/E^M$ is isomorphic to $X/F$).
	\end{prop}
	\begin{proof}
		$S_\alpha(\emptyset)$ is an uncountable Polish space. Therefore, there is a Borel isomorphism $\Phi\colon S_\alpha(\emptyset)\to X$. We can then define $\mathrel{E}$ as 
		\[
		a\mathrel{E}b\iff \Phi(\tp(a/\emptyset)) \mathrel{F} \Phi(\tp(b/\emptyset))
		\]
		Since $M$ is atomic, each complete $\emptyset$-type extends uniquely to an $M$-type, so $\Phi$ induces an isomorphism between $S(M)/E^M$ and $X/F$.
	\end{proof}
	
	\begin{thm}
		Let $I\unlhd 2^{\bf N}$ be a Borel ideal. Then the equivalence relation $E_I$ can be realised (in the sense of Borel cardinality) as a relation $E$ which refines $\equiv$, is refined by $\equiv_{\KP}$, is orbital, and is only defined on a single complete type.
	\end{thm}
	
	\begin{prop}
		Any Borel cardinality can be realised by an equivalence relation $E$ which is refined by $\equiv$.
	\end{prop}
	
	\begin{prop}
		Suppose $E_n$ are bounded,  Borel equivalence relations on some type-definable sets $X_n$ (in some theories $T_n$, possibly all distinct). Then there is $E\sim_B \prod_nE_n$ which is also bounded, Borel equivalence relation in some theory $T$.
		
		Furthermore, the following properties are preserved by $E$ (i.e.\, if all $E_n$ have this property, then so does $E$):
		\begin{itemize}
			\item
			refining or being refined by $\equiv$
			\item
			refining or being refined by $\equiv_{\KP}$
			\item
			being orbital
			\item
			being orbital on types
			\item
			being defined on a single complete type
		\end{itemize}
	\end{prop}
	
	\begin{prop}
		Suppose $E_n$ are Borel equivalence relations are realised by some bounded, Borel equivalence relations on single types $p_n$ (in some theories $T_n$, possibly all distinct). Then there is $E\sim_B\prod_nE_n/\Fin$ which is also a bounded, Borel equivalence relation on a single type $p$ in some theory $T$.
		
		Furthermore, the following properties are preserved by $E$ (i.e.\, if all $E_n$ have this property, then so does $E$):
		\begin{itemize}
			\item
			being refined by $\equiv_{\KP}$
			\item
			being orbital
		\end{itemize}
	\end{prop}
	
	\begin{prop}
		Suppose $E_n$ are the restrictions to a single ${\equiv_{\KP}}$-class of bounded, Borel equivalence relations (in some theories $T_n$, possibly all distinct). Then there is an $E\sim\prod_nE_n/\Fin$ which is also realised as a restriction to a single ${\equiv_{\KP}}$-class of a bounded, Borel equivalence relation in some theory $T$.
		
		Furthermore, if $E_n$ are restrictions of orbital equivalence relations, then so is $E$
	\end{prop}
	
	(Countable unions, countable intersections!)
	
	The picture we get is as follows.
	
	For arbitrary bounded and Borel equivalence relations $E$ we have:
	\begin{itemize}
		\item
		If we allow $E$ to be refined by $\equiv$, then $E$ can have any Borel cardinality at all.
		\item
		If $E$ equal to (a restriction to a type-definable set of) $\equiv$ or $\equiv_{\KP}$, then $E$ must be smooth (and can be any smooth relation can be obtained this way).
		\item
		The class of possible Borel cardinalities for $E$ defined on a single type and refined $\equiv_{\KP}$ contains (equivalence classes of) all smooth equivalence relations, all $E_I$ for Borel ideals $I$ (like ${\bf E}_0,{\bf E}_1,{\bf E}_2,{\bf E}_3$), and it is closed under countable products and Fubini products (i.e.\ $\prod_n/\Fin$).
		\item
		The class of possible Borel cardinalities for restriction of $E$ of a single $\equiv_{\KP}$-class contains no smooth class except the trivial one, but it does contain $\ell^\infty$ and ${\bf E}_0$, and it is closed under countable products and Fubini products.
		\item
		If $E$ is the restriction of ${\equiv_{\Lasc}}$ to a type-definable set, then $E$ may be smooth (in which case it is equal to $\equiv_{\KP}$), or it may have the Borel cardinality of either ${\bf E}_0$ or $\ell^\infty$, and it is not known if it can have any others, and the same is true about the Borel cardinality of the restriction of $E$ to a single ${\equiv_{\KP}}$ class.
		\item
		All points above, except for the first one, remain true if we add the assumption that $E$ is orbital. It is not known whether there is some Borel cardinality which can only be attained by a non-orbital $E$.
	\end{itemize}
	
	\begin{qu}
		Suppose $E$ is a bounded, Borel equivalence relation defined on a (set of realisations of) a single complete type $p$. If $E$ refines $\equiv_{\KP}$ and it is not smooth, then is the Borel cardinality of $E$ the same as the Borel cardinality of $E$ restricted to a single ${\equiv_{\KP}}$ class?
	\end{qu}
	
	
	
	\section{Every quotient of a Polish group is a strong type}
	\begin{prop}
		Suppose $G$ is a pro-(definable group) in $T$, while $H\leq G$ is invariant of bounded index.
		
		Then there is a reduct $T'$ of $T$, a complete $\emptyset$-type $p$ (in $T'$) and strong type $E$ on $p$, such that for every small $M\models T'$ we have an $M$-type-definable bijection $f\colon G\to p(\fC')$ such that $f^{-1}[E]$ is the left coset equivalence relation of $H$ on $G$.
	\end{prop}
	
	\begin{fct}
		Every compact Hausdorff group is a pro-(compact lie group), i.e.\ it is the inverse limit of a surjective system of compact Lie groups. Moreover, we can choose the system in such a way that the connecting maps are analytic.
	\end{fct}
	
	\begin{dfn}
		A \emph{restricted analytic function} is a function $f\colon [0,1]^n\to \bR$ which is analytic on an open set containing $[0,1]^n$.\xqed{\lozenge}
	\end{dfn}
	
	\begin{fct}
		Suppose $T$ has NIP and $G$ is a definable group. Then for any small $A$ we have $G^{0}_A=G^{0}_\emptyset$, $G^{00}_A=G^{00}_\emptyset$ and $G^{000}_A=G^{000}_\emptyset$. In particular, $G^0, G^{00}$ and $G^{000}$ exist.
	\end{fct}
	
		
	\begin{fct}
		The theory $T=\Th(\bR_{\an})$ of the real field expanded with all restricted analytic functions is o-minimal. In particular, it has NIP.
	\end{fct}

	\begin{fct}
		Suppose $G$ is group definable, which is definable in an o-minimal expansion of the real field. Suppose in addition that $G(\bR)=G_\bR$ is a compact Lie group. Then $G/G^{00}=G_{\bR}$.
	\end{fct}
	
	\begin{prop}
		Let $T=\Th(\bR_{\an})$ be the theory of the real field expanded with all restricted analytic functions. Then for every compact Hausdorff group $G_{\bR}$, there is a pro-(definable group) $G$ such that $G/G^{00}$ is isomorphic to $G_{\bR}$ (as a topological group).
	\end{prop}
	\begin{proof}
		Choose a compact Hausdorff group $G_{\bR}$ and express it as the inverse limit $\varprojlim_i G_{\bR,i}$, where $G_{\bR,i}$ is a compact Lie group. Then clearly $G_{\bR,i}$ is definable in $\bR$, and the connecting maps are analytic, and let $G=\varprojlim_i G_i$. Then $G/G^{00}=\varprojlim_i G_i/G_i^{00}=\varprojlim_i G_{\bR,i}=G_{\bR}$.
	\end{proof}
%	\printbibliography
\end{document}

